\documentclass[10pt,letterpaper]{article}
\begin{document}
\title{DETECTION AND RECOVERY LAPTOPS AND MOBILE DEVICES IN UGANDA.}
\author{by .Ntambi Isaac   \\ 216014342 \\  16/U/10485/PS}
\maketitle
\section {Abstract }
The issue of dropout rates within minority groups has become an increasingly important problem. This is especially true for those who live in recognized at-risk communities. There is little present literature that speaks to the specific factors that affect East-African youth and their reasoning behind dropping out of school. This study attempts to analyze the differences through using interviews from knowledgeable educators and available literature/studies to present the main factors contributing to this issue and what solutions can be developed to solve it.
\section{Introduction }
In several areas of Uganda theft is on a climax due to increased unemployment and drop out. Among the items or valuables that are being lost include laptops of all categories like HP, Dell, Toshiba and so on.And among the other devices we have smartphones of all brands android and Windows versions. These devices are being lost by the rightful owners especially around urban areas . Once lost they are never seen again by the rightful owners and this going rampant.

\section{Statement of problem. }
The major purpose of this research is to devise ways of detecting and recovering lost laptops and mobile devices in Uganda.

\section{Significance of study.}
The study is to benefit the rightful owners of several mentioned devices to detect and recover them . As some of these devices are so expensive acquire new ones once lost.
\section{Scope of study.}
This research is encompassed to span this act in Uganda and to start with the urban areas where there is high population and more prone to theft.
.
\section{Methods of study.}
Source of data. The information pertaining this research was gathered from several points and in various ways. The data collection was done by me as I approached several individuals with well built up questions that helped me gather the required information. This was done randomly from various places of the country.
\section{Limitations of study.}

As one the ideas of collecting data was use of questionnaires and these limit the information gathered as some of the questions that might be of need may not be added to the questionnaire.
Findings.
According to an approximate value of people losing laptops and mobile devices is 20 percent everyday in several parts of Uganda.
\section{Conclusion..}
According to the survey many people are negatively affected by this act as they have to minimise on their expenditure to ensure new gadgets as replacement for the lost ones.
\section{Recommendations..}
The individuals using these devices are to be introduced to the new ideas of detecting and recovering their devices as way to avoid rebuying , this would be in terms of GPS devices that are be inbuilt or inserted in the new devices before use.
\end{document}